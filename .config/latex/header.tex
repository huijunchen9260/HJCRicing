%\documentclass{article} %虽然加了注释号,但请注意这一行绝对不能注释掉!因为pandoc后生成的tex文件已含有此句
\usepackage{fontspec}  %加這個就可以設定字體
\usepackage{xeCJK}
\setmainfont[Scale=1.2]{TeX Gyre Termes}
\setCJKmainfont[Scale=1.2]{WenQuanYi Zen Hei}
\renewcommand{\baselinestretch}{1.1}
\CJKsetecglue{} %設定中英混雜時不會讓英文後面多空一格
\XeTeXlinebreaklocale "zh"
\XeTeXlinebreakskip = 0pt plus 1pt %這兩行一定要加,中文才能自動換行
\usepackage{parskip}
\setlength{\parindent}{0in} %讓開頭不擡頭
\setlength{\parskip}{1em} %讓空一行就等於分段

\usepackage[pagestyles]{titlesec} %讓我可以利用下方的程式碼來自訂section等等的格式

%%%%%%%%下方的程式碼可以自定
%\titleformat{\part}{\Huge\scshape\filcenter}{}{1em}{}
\titleformat{\section}{\LARGE\bfseries\raggedright}{\thesection}{1em}{}[{\titlerule[1pt]}]
\titlespacing{\section}{0pt}{2.5em}{1em}
\titleformat{\subsection}{\Large\bfseries\itshape\raggedright}{\thesubsection}{1em}{\underline}%[\rule{3cm}{.2pt}]
\titlespacing{\subsection}{0pt}{2em}{0.5em}
\titleformat{\subsubsection}{\large\bfseries\itshape\raggedright}{\thesubsubsection}{1em}{}%[\rule{3cm}{.2pt}]
\titlespacing{\subsubsection}{0pt}{1.5em}{0.5em}

%%%%%%%%上方的程式碼可以自定
